\section{Conclusion}

%It is difficult to overstate the importance of the role that intuition plays in guiding research. While it does not appear within a rigorous experiment, it nevertheless frames it. Intuition attracts researchers to take interest in some things and not others. Intuition prompts researchers to pose the questions that are ultimately refined into testable hypotheses. Intuition guides the priorities researchers apply when they decide which hypothesis is worthy of rigorous testing. Good intuition is as much a part of the scientific process as the careful logical reasoning that ultimately sustains or rejects the hypothesis. The difficulty is that intuition is heuristic, and heuristics draw on experience. The great majority of the complexity to be found in microbial ecology exists outside of what can be experienced directly, and so researchers often struggle to frame their work.

%The approach we present here does not directly test the interactions it predicts. The predictions emerge in much the way a observer would draw inferences regarding an unfamiliar ecosystem from their experience with other systems. In this way, we may begin to see microbial ecosystems as part of the world we already know.

Ascertaining the ecological function of organisms in the microbiome is a critical step along the path to understanding how microbiomes work. For relationships that have persisted long enough to shape the evolution of the host and the microbe, patterns in the evolution of the host and the microbe may provide insight into the type of mutual adaptation operating in the relationship. This information, if it is present, will be embodied in the topology of the phylogenetic trees of the two groups of interacting organisms. Graph theory provides a sophisticated suite of tools for interrogating the topology of trees, but not a framework for assaying the significance of the features it illuminates. 

Rather than constructing an implicit or explicit model of how we expect coevolution to influence the topology of the host and microbial phylogenies, we propose a comparative approach. The graph theoretic view of tree topology will admit generalized graphs without any difficulty, and so one may construct graph objects that represent complete ecological relationships in their evolutionary context. Graph theory furnishes a dissimilarity metric, as well as a way of extracting moments on topological features, that serve as the basis of a feature space in which the topology of graphs are projected. In this feature space, one may project graphs for which the nature of the ecological relationship shaping the interaction is understood. These labeled interactions can then be used to train and evaluate classifiers, which can then be used to make predictions about the nature ecological relationships of interactions that are unknown.

\subsection{Future directions}

We demonstrate this process using a small dataset of microbiome data from a group of 14 host organisms and a neural network trained on small collection of 51 labeled interactions. While the results are promising, we stress that this is a proof-of-concept. A much larger collection of labeled interactions would be preferable for training the neural network, and would make it possible to conduct a rigorous evaluation of the error rate. A more comprehensive collection of host organisms could be subdivided into different adaptive innovations, opening up the possibility of identifying key microbial players in, for example, the evolution of trophic strategies. In this study, we use a region of the 16S rRNA gene to reconstruct the microbial phylogeny. While this choice maximizes the microbial diversity we are able to observe, the slow evolution of this gene limits the sensitivity of our proof-of-concept study to interactions that take place on long time scales. A metagenomic approach would make it possible to target more quickly evolving genes and specific functions. Lastly, and perhaps most interestingly, we note that the generalization to a graph structure is capable of representing much more than binary ecological interactions. It would be trivial to incorporate three trees representing the evolution and ranges of a host, a parasite and a vector. With an arbitrary number of trees, one could use this framework as a phylogenomic tool for classifying coalescent events by the patterns left in the way orthologs and paralogs assort within and among genomes. 