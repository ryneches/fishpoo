
\begin{figure}
    \centering
    \begin{subfigure}[b]{0.45\textwidth}
        \includegraphics[width=3.5in]{FishPoo/figures/phylosig_heattree_highsig}
        \small
        \caption{{\tt CHALBRI1\_1198}}
    \end{subfigure}
    \begin{subfigure}[b]{0.45\textwidth}
        \includegraphics[width=3.5in]{FishPoo/figures/phylosig_heattree_lowsig}
        \small
        \caption{{\tt CHALBRI1\_4931}}
    \end{subfigure}\\
    \begin{subfigure}[b]{\textwidth}
        \resizebox{\textwidth}{!}{%
        \begin{tabular}{@{}lllcccccc@{}}
            \toprule
            \textbf{OTU name} & \textbf{Tax. assignment} & \textbf{Tax. level} & \textbf{\% id.} & \textbf{Rel. abund.} & \textbf{Bloomberg's $K$} & \textbf{Pagel's $\lambda$} & \textbf{Moran's $\hat{I}$} & \textbf{Abouheif's $C_{\mathrm{mean}}$} \\ \midrule
            {\tt CHALBRI1\_1198} & {\em Aeromonas} & genus & 99\% & 0.14\% & 0.792551 & 0.640501 & 0.330350 & 0.394249 \\
            {\tt CHALBRI1\_4931} & Enterobacteriaceae & family & 100\% & 0.12\% & 0.425427 & 0.000047 & 0.005716 & 0.046999 \\
            \bottomrule
        \end{tabular}%
        }
    \end{subfigure}
    \caption{Two examples of bacterial associations examined as traits of their hosts, one with high phylogenetic signal \textbf{(a)} and one with low phylogenetic signal \textbf{(b)} (trait values in the tree legends are expressed in parts per hundred).}
    \label{fig:FP_fig4}
\end{figure}