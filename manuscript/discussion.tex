\section{Discussion and Results}

Identification of ecological function of organisms or groups of organisms in microbial communities, particularly host-associated microbiomes, is a key goal for many avenues of theoretical and applied research. Broadly speaking, there are three ways to approach the question.

Ecological function can be predicted by correlating the relative or absolute abundance of microbial groups with a host response or environmental outcome. For many systems, this is the only approach available. In clinical applications, all hosts belong to a single species ({\em H. sapiens}) exhibiting very little genomic diversity. For this reason, correlation-based studies in humans, such as Genome Wide Association Studies (GWAS), usually require a very large sample size to achieve acceptable false discovery rates. \cite{pearson2008interpret} For marine communities, the cosmopolitan distribution of most organisms requires a planetary perspective on niche occupancy. \cite{jiang2012functional,violle2014emergence} While these approaches are powerful research tools, correlations alone cannot establish the ecological function of a organism.

Ecological function can be predicted by examining the impact of an organism on the fitness of its host. Of course, this requires both a meaningful definition of host fitness and a reliable way to measure it. \cite{kutzer2016maximising,rynkiewicz2015ecosystem}

Ecological function can be predicted by conceptually treating a host-associated organism as a trait of the host, and examining how the ``trait'' assorts with the evolution of other host phenotypes. \cite{easson2014phylogenetic,schottner2013relationships} This phylogenetic signal approach \cite{felsenstein1985phylogenies,revell2008phylogenetic,munkemuller2012measure} is very powerful, but in practice is can be confounded by the extraordinary nuance of microbial diversity. Very closely related microbial organisms can have radically divergent ecological functions -- for example, {\em Escherichia coli} Nissle 1917 is evidently beneficial to the host, \cite{gronbach2010safety} while {\em Escherichia coli} O157:H7 is a deadly pathogen. \cite{gally2017microbe} It is not always apparent where to draw the boundary between two such ``traits,'' nor is it always straightforward to determine which side of the boundary an observation falls. This is not to say that the microbial-relationship-as-host-trait model is not useful, but rather that it is useful for exploring relationships that have a well-characterized scope with respect to the diversity of the microbial component. Its usefulness for discovering new relationships is limited.

Ecological function can be predicted by the likelihood of coevolution between a group of hosts and a group of microbes. \cite{hafner1994disparate,hommola2009permutation} This approach is limited by the fact that coevolution can result from different and perhaps antithetical ecological scenarios. \cite{van1973new,janzen1980coevolution,bergstrom2003red}

By placing unknown interactions in the microbiome into context with the evolution and ecology of characterized interactions, we overcome most of these difficulties. Because the effect of the relationship is implicit in the characterization of labeled interactions, we do not draw {\em post hoc} conclusions from correlations with outcomes. The effect of the relationship on host fitness is implicit in the phylogeny of both organisms. The scope of microbial diversity is explicitly addressed by the use of the microbial phylogeny. If the phylogeny fails to capture the necessary scope, the method fails gracefully, yielding an inconclusive result. Finally, by directly addressing the underlying driver of coevolution, we can make positive predictions about these mechanisms.

All model-based methods must convincingly demonstrate that they provide the appropriate selectivity, sensitivity and parameter selection for their application. This is more challenging for some applications than others, and searching for coevolution in microbiomes appears to be particularly challenging. Comparative methods exchange the statistical vulnerabilities of model-based methods for the epistemological vulnerabilities of database bias. Fortunately, the number of multi-species interactions is combinatorial with the number of species (though constrained by propinquity), and the ecological literature is rich with examples. A database drawn from the existing literature would suffer from bias, but it would not be small.

Rather than comparing candidate cases to a model, a comparative approach calls for a metric that scales with dissimilarity among cases. Measures of dissimilarity are not summary statistics (avoiding the problem of supervision), and it is possible to construct them with fewer assumptions. The construction of a feature space of topological properties of interactions and dissimilarities with respect to interactions of known ecologies makes it possible to cast the question of the ecological function as a machine learning problem. This sketches out a powerful and flexible framework for extracting inferences on the nature of many kinds of ecological interactions without direct observation of their mechanism. The cost is that one must assemble a collection of relevant training data, and it is limited to cases where the interaction has persisted long enough to leave an significant imprint in the evolutionary history of the interacting groups. It should work particularly well in cases where two or more adaptive radiations have interacted.

The training problem can be visualized by selecting a subspace spanned by two axes of the feature space and projecting the labeled training data and the unlabeled experimental data into it. Alternatively, one can project into a subspace spanned by principle components. Similarly, the predictions can be visualized by projecting the experimental data into one of these subspaces with labels corresponding to the predictions (Figure \ref{fig:FP_classified}).

\subfile{figures/figure13_subfig}

This study is limited by the small number of labeled interactions we were able to extract from the literature (51 interactions from the literature and 100 simulated interactions). With a training set of suitable size, the machine learning process calls for the refinement of the classifier by splitting the training data into training and testing sets. The classifier should be trained using the training set and its performance scored using the testing set. The tuning parameters of the classifier would be adjusted, re-trained and re-scored using multidimensional gradient descent to minimize the classification error. Here, we show only a single iteration of this process using a trained but unoptimized neural network as our classifier. Our trained neural network predicts the correct labels for the interactions with which it was trained about 97-98\% of the time (there is some stochastic variation), but this does not represent a rigorous examination of its accuracy on unlabeled data. Fortunately, our training set represents a negligible fraction of the interactions found in the ecology literature. The effort of extracting, reformatting and standardizing it is the only thing that limited us to 50 examples. An automated approach is under development, but is beyond the scope of this study.
